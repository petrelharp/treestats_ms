\documentclass[12pt,letterpaper]{article}
% Modified from the following by peter ralph, 11/4/2013

\usepackage{graphicx}
\usepackage{ifthen}
\RequirePackage{times}

%%%%%%%%%% EXACT 1in MARGINS %%%%%%%                                   %%
\setlength{\textwidth}{6.5in}     %%                                   %%
\setlength{\oddsidemargin}{0in}   %% (It is recommended that you       %%
\setlength{\evensidemargin}{0in}  %%  not change these parameters,     %%
\setlength{\textheight}{8.8in}    %%  at the risk of having your       %%
\setlength{\topmargin}{0in}       %%  proposal dismissed on the basis  %%
\setlength{\headheight}{0in}      %%  of incorrect formatting!!!)      %%
\setlength{\headsep}{0in}         %%                                   %%
\setlength{\footskip}{.2in}       %%                                   %%
%%%%%%%%%%%%%%%%%%%%%%%%%%%%%%%%%%%%                                   %%


\setlength{\parskip}{1em plus4mm minus3mm}

\newcommand*{\name}[1]{\def\fromname{#1}}
\newcommand*{\signature}[1]{\def\fromsig{#1}}
\newcommand*{\address}[1]{\def\fromaddress{#1}}
\newcommand*{\location}[1]{\def\fromlocation{#1}}
\newcommand*{\telephone}[1]{\def\telephonenum{#1}}
\name{}
\signature{}
\address{}
\location{}
\telephone{}

\newcommand{\ucluniv}{UO}
\newcommand{\uclcampus}{}
\newcommand{\uclschool}{Departments of Mathematics and Biology}
\newcommand{\uclunivstreet}{335 Pacific Hall, University of Oregon}
\newcommand{\uclunivtown}{Eugene, OR 97403-5289}
\newcommand{\ucldept}{Institute for Ecology and Evolution}
\newcommand{\ucldeptphone}{(541) 346-4532}
\newcommand{\ucldeptfax}{}
\newcommand{\ucldepturl}{http://ie2.uoregon.edu/}
\newcommand{\ucltelephone}{}
\newcommand{\uclemail}{}
\newcommand{\uclfax}{}
\newcommand{\dept}[1]{\renewcommand{\ucldept}{#1}}

% These commands show up in the return address, but NOT in the footer on each page
\newcommand{\email}[1]{\renewcommand{\uclemail}{#1}}
\newcommand{\fax}[1]{\renewcommand{\uclfax}{#1}}


\renewcommand{\telephone}[1]{\renewcommand{\ucltelephone}{#1}}
\newcommand{\sixrm}{\fontfamily{ptm}\fontsize{6}{8}\selectfont}
\newcommand{\svnrm}{\fontfamily{ptm}\fontsize{7}{9}\selectfont}
\newcommand{\egtrm}{\fontfamily{ptm}\fontsize{8}{10}\selectfont}
\newcommand{\twlrm}{\fontfamily{ptm}\fontsize{12}{14}\selectfont}
\newcommand{\ctrdot}{\hspace*{1.4pt}$\cdot$\hspace*{1.4pt}}

\newcommand{\uscaddress}{
  \begin{minipage}[t]{2.0in}
      \vspace{.2em}
      \ifthenelse{\equal{\fromname}{}}{}{{\sc \fromname}\\}
      \ifthenelse{\equal{\ucltelephone}{}}{}{{\sc Phone:} \ucltelephone\\}
      \ifthenelse{\equal{\uclfax}{}}{}{{\sc Fax:} \uclfax\\}
      \ifthenelse{\equal{\uclemail}{}}{}{{\sc Email:} \uclemail\\}
  \end{minipage}
}


\newsavebox{\uscletterhead}
\sbox{\uscletterhead}{\parbox[t]{6.5in}{
    \parbox{2.925in}{\includegraphics{uo-logo}}
        \begin{minipage}{3.975in}
          \hfill
          \begin{minipage}{3in}
            \egtrm\MakeUppercase
            \ucldept \\
            \ifthenelse{\equal{\uclunivstreet}{}}{}{
          \egtrm\MakeUppercase \uclunivstreet \\
          }
          \egtrm\MakeUppercase \uclunivtown \\
            \ifthenelse{\equal{\ucldeptphone}{}}{}{\hspace\fill\ucldeptphone} 
      \end{minipage}
    \end{minipage}
    \vspace{0.5em}
    \hrule
  }
}


\newdimen\longindentation
\longindentation=.5\textwidth
\newdimen\indentedwidth
\indentedwidth=\textwidth
\advance\indentedwidth -\longindentation

\newcommand{\firstpage} { \centerline{\parbox{6.5in}{\usebox{\uscletterhead}}} }

\newcommand{\closing}[1]{\par\nobreak\vspace{\parskip}%
  \noindent
  \ifx\empty\fromaddress\else
  \hspace*{\longindentation}\fi
  \parbox{\indentedwidth}{\raggedright
       \ignorespaces #1\\[1\medskipamount]%
       \ifx\empty\fromsig
           \fromname
       \else \fromsig \fi\strut}%
   \par}

\newcommand{\opening}[1]{
  \firstpage
  \vspace*{.05in}
    \noindent\rule{4.25in}{0mm}
    \uscaddress
  \vspace{1ex}

  \noindent #1
  \vspace{1ex}
}


\usepackage[hidelinks]{hyperref}


\signature{
    Peter Ralph, on behalf of Kevin Thornton and Jerome Kelleher.}
\name{Peter Ralph}
\address{Peter Ralph\\ Institute for Ecology and Evolution\\ and Departments of Mathematics and Biology,\\ University of Oregon\\
plr@uoregon.edu}
\telephone{(707) 502-5854}
\email{plr@uoregon.edu}

\begin{document}

\pagestyle{plain}

\noindent 
  

\opening{To the editor(s) --}
\hspace{\stretch{1}}
\today

    We are writing to submit our manuscript, entitled
    \emph{Efficiently summarizing relationships in large samples: a general
        duality between statistics of genealogies and genomes},
    for consideration for publication in \textit{Genetics}.

    An important part of the paper is that it introduces fast methods for computation
    of population genetic statistics from large numbers of genomes using tree sequences,
    available through the tested and validated python package \texttt{tskit}.
    However, that's not the main point ---
    if it was, we'd be submitting to a more bioinformatics-focused journal. 
    It is well-known that population genetics statistics say something about the underlying genealogies,
    although it is not common for researchers to explicitly discuss this.
    The bigger point of our manuscript is to describe a general framework that makes it clear
    precisely what aspect of the underlying genealogies is described by each statistic,
    under the infinite-sites model of mutation,
    and to provide methods to compute these statistics of tree shape as well.
    We think this makes population genetics statistics conceptually cleaner,
    but has some other possible benefits, besides speed:
    (1) we make it easy to invent and implement new statistics;
    (2) the framework allows separation of mutational from demographic noise;
    and (3) if unbiased tree sequences can be inferred from genomic data,
    summarizing the trees themselves provides a descriptive method
    insensitive to variation in substitution rate or perhaps even marker ascertainment.
    The last is dependent on testing and improvement of tree sequence inference,
    but would be tremendously useful, if it works out.

    We have put a lot of effort into making the paper accessible to the readership of \textit{Genetics}.
    The paper begins with a careful treatment of the theoretical framework and many examples,
    then illustrates points with simulation and a tree sequence inferred from 1000 Genomes data
    (by Speidel et al, 2019),
    and finally describes the fundamental algorithm and compares computational efficiency
    to other methods.
    The definitions and framework might seem lengthy,
    but in our assessment they are necessarily so:
    there are many false paths that intuition can lead even experienced theoreticians down,
    and choices we have made in our framework that might seem strange at first
    turn out to be natural once carefully considered.

    We are enthusiastic about the potential impact of this paper
    and its interest to the readers of \textit{Genetics}.
    We hope that you agree.

\closing{Sincerely,}


\end{document}

