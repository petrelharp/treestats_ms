\documentclass{article}
\usepackage{fullpage}
\usepackage{graphicx}

\usepackage{color}
\usepackage{amsmath,amssymb,amsthm}
\usepackage{natbib}

\bibliographystyle{plainnat}


\newcommand{\R}{\mathbb{R}}
\newcommand{\Q}{\mathbb{Q}}
\newcommand{\Z}{\mathbb{Z}}
\newcommand{\N}{\mathbb{N}}

\renewcommand{\P}{\mathbb{P}}
\newcommand{\E}{\mathbb{E}}
\newcommand{\var}{\mathop{\mbox{Var}}}
\newcommand{\cov}{\mathop{\mbox{cov}}}
%\newcommand{\det}{\mathop{\mbox{det}}}
\newcommand{\supp}{\mathop{\mbox{supp}}}
\newcommand{\sgn}{\mathop{\mbox{sgn}}}

\newcommand{\bone}{\mathbf{1}}
\newcommand{\st}{\,\colon\,}

% These macros are borrowed from TAOCPMAC.tex
\newcommand{\slug}{\hbox{\kern1.5pt\vrule width2.5pt height6pt depth1.5pt\kern1.5pt}}
\def\xskip{\hskip 7pt plus 3pt minus 4pt}
\newdimen\algindent
\newif\ifitempar \itempartrue % normally true unless briefly set false
\def\algindentset#1{\setbox0\hbox{{\bf #1.\kern.25em}}\algindent=\wd0\relax}
\def\algbegin #1 #2{\algindentset{#21}\alg #1 #2} % when steps all have 1 digit
\def\aalgbegin #1 #2{\algindentset{#211}\alg #1 #2} % when 10 or more steps
\def\alg#1(#2). {\medbreak % Usage: \algbegin Algorithm A (algname). This...
  \noindent{\bf#1}({\it#2\/}).\xskip\ignorespaces}
\def\kalgstep#1.{\ifitempar\smallskip\noindent\else\itempartrue
   \hskip-\parindent\fi
   \hbox to\algindent{\bf\hfil #1.\kern.25em}%
   \hangindent=\algindent\hangafter=1\ignorespaces}

\newcommand{\algstep}[3]{\kalgstep #1 [#2] #3 }
\newenvironment{taocpalg}[3]{%
\vspace{1em}%
\algbegin Algorithm #1. ({#2}). #3 }
{\vspace{1em}}

\newcommand{\algref}[1]{#1}


\newtheorem{definition}{Definition}
\newtheorem{lemma}{Lemma}
\newtheorem{cor}{Corollary}
\newtheorem{theorem}{Theorem}
\newtheorem{example}{Example}



\newcommand{\tskit}{{\texttt{tskit}}}
\newcommand{\branch}{\mbox{Branch}} % branch stat
\newcommand{\branchp}{\mbox{Branch}_+} % polarized
\newcommand{\site}{\mbox{Site}} % site stat
\newcommand{\sitep}{\mbox{Site}_+} % polarized
\newcommand{\node}{\mbox{Node}} % node stat
\newcommand{\nodep}{\mbox{Node}_+} % polarized

\newcommand{\treeseq}{\mathbb{T}} % tree sequence
\newcommand{\iw}{w} % sample (initial) weights
\newcommand{\tiw}{w_\text{total}} % total sample (initial) weights
\newcommand{\nw}{x} % subtree (node) weights
\newcommand{\aw}{{\bar x}} % allele weights

\newcommand{\plr}[1]{{\color{blue}\textbf{plr:} \it #1}}
\newcommand{\jk}[1]{{\color{red}\textbf{jk:} \it #1}}


\begin{document}

\begin{center}
    Efficiently summarising relationships in large samples:
    a general duality between statistics of genealogies and genomes
\end{center}

\emph{Running head:} Genealogies and genomes


%%%%%%%%%%
\paragraph{Abstract}
Here we describe how to efficiently compute single-site statistics
from population genetic data using the succinct tree sequence
encoding for correlated genealogies.
The method is general enough to compute any currently-defined statistic,
and we use this general framework to suggest a number of new statistics.
\plr{say what method is: sthg about weighting fns}
This formulation also makes it easy to see what statistic of the underlying genealogical trees
each sequence-based statistic is estimating,
and, if underlying trees are also available,
the same methods can be used to quickly compute these quantities.


%%% OUTLINE
% 1. Framework and algorithms for computing general stats
%
%     * Framework:
%
%         - single-site stats are functions of the genotype vector
%         - in infinite sites, the genotype vector of a mutation is determined by the nodes below it
%         - ex: divergence as a sum over trees and branches
%         - ex: ancestry proportions
%         - general definition, using summary function of weights propagated up the tree
%             * recursion relating weights on nodes of two forests that differ by a branch
%
%     * Review of tree sequences (quick)
%
%     * Algorithms:
%
%         - weight propagation across tree differences
%         - count by node length ("Node Statistics")
%         - count by branch area ("Branch Statistics")
%         - count by site ("Site Statistics")
%         - note on outputs (by node, by site, in windows)
%
% 2. Examples: in sections, one of each type of output
%
%     - ancestry proportions
%     - list of standard summary statistics: divergence, Fst, f4, covariance
%         * performance for one of these
%     - PCA: Krylov method
%         * performance
%
% 3. Correspondence to tree statistics
%
%     - math: in infinite sites, neutral model, E[site] = branch
%     - plot from simulation showing branch stats are less noisy versions of the site stats
%     - show time profile of PC1 as computed from rare variants and common variants



%%%%%%%%%%%%%%%%%%%%%%%
\section*{Introduction}

\plr{need something different here}
Today's vast quantity of whole-genome sequence
makes it possible to confidently measure quantities
between \emph{individuals} that previously had to be roughly estimated
averaging over populations.
For instance, XXX estimated $F_{ST}$ between XX from allozyme sequence,
% JK: FIXME what's the slatkin ref here?
but now we know that $F_{ST}$ can be viewed as $1 - t_{within}/t_{total}$ \citep{slatkin_fst}
which can therefore be estimated precisely by $1 - \pi_{within}/\pi_{total}$.

% {computational challenges}
Computation is beginning to be a major problem:
whole-genome sequence on the scale of the UK Biobank,
for instance, might have 500K individuals genotyped at 10M loci (??),
resulting in a genotype matrix of $5 \times 10^{12}$ entries.
Computing a single summary from this matrix is still relatively quick,
once loaded into memory,
but for many applications we are now interested in computing a large number of statistics
(for instance, consider the roughly $10^{11}$ different possible pairwise divergences).
Indeed, many sorts of modern inference machinery, such as deep learning,
depend on being able to compute enough statistics that this becomes a serious bottleneck in practice.

The \emph{succinct tree sequence} (or tree sequence, for brevity) is a data
structure that efficiently encodes the genealogical trees describing how a set of
individuals are related to each other at each point along their genome. The
data structure was introduced in the context of coalescent
simulation~\citep{kelleher2016efficient}, and lead to scalability increases of
several orders of magnitude over existing methods. It was subsequently extended
and refined for forwards-time simulations~\citep{kelleher2018efficient,haller2018tree}
where similar efficiency gains were made. Recent
work~\citep{kelleher2018inferring} has shown that tree sequence algorithms can
also be used to massively increase the scalability of methods for inferring
genome-wide genealogies, and shown that it is feasible to infer trees for
millions of whole genome samples.

The key to the remarkable scalability and efficiency of tree sequence
algorithms is the way in which the shared structure in adjacent trees along
the genome is encoded. As one looks across the genome the genealogies change
because of the results of recombination in ancestors.
However, nearby trees tend to share a lot of common structure,
and single genealogical relationships (e.g., individual $x$ inherits from individual $y$)
are often shared across relatively long distances,
which manifest as shared \emph{edges} across many adjacent trees in the tree sequence.
This redundancy can be used
to both store genome sequence and the associated genealogical relationships extremely compactly,
as well as to compute statistics highly efficiently~\citep{kelleher2016efficient}.
This paper generalizes the latter strategy to a much broader class of statistics.

\plr{Review of tree shape stuff in phylogenetics}
a good review of theoretical properties \citet{semple2003phylogenetics}.

% summary
In this paper, we present a general framework for defining and efficiently computing
``single-site'' genetic statistics,
i.e., statistics of aligned genome sequence that can be expressed as averages over values computed
separately for each site.
Patterns of genetic polymorphism in a population
are determined by the population's history of Mendelian inheritance,
as summarized by the underlying genealogical trees,
and so it is natural (and helpful) to express these statistics
in terms of the structure of the trees themselves.
This perspective turns out to both suggest efficient algorithms
for computing these statistics on tree sequences,
and allows us to treat single-site genetic statistics in a more general framework
of summaries of tree shape and inheritance.
The methods are implemented in the \tskit{} library.

%%%%%%%%%%%%%%%%%%%%%%%
\section*{Framework and statistics}


Suppose we have sequenced a linear genome of $L$ bases from $n$ (homologous) sampled chromosomes,
and wish to compute a statistic of the allele frequency spectrum.
In other words, if the frequency of allele $a$ at a site $i$ is $p_i(a)$,
and the allele frequency spectrum
-- i.e., the proportion of alleles whose frequency is $p$ in our sample -- is $h(p)$,
then for some function $f()$, we want to compute $\sum_p h(p) f(p)$.
Since this is a simple average over sites, and monomorphic sites do not contribute,
we can compute this, summing over the $L$ loci, as
$$\sum_p h(p) f(p) = \frac{1}{L} \sum_{i=1}^L \sum_a f(p_i(a)).$$
(Note that since the ancestral allele is included, this is a function of the
\emph{folded} frequency spectrum, so is invariant to relabeling of alleles.)
If we know where each mutation has occurred on the genealogy,
we can easily find the frequency of the corresponding allele by counting how many samples
occur below that mutation in the tree (and accounting for those that subsequently mutate).
What we do here is a generalization of this idea.


Roughly speaking, we will
(a) propagate ``weights'' additively up each tree,
(b) summarize these weights on each branch with a ``summary function'',
and (c) aggregate these summaries to produce statistics
(e.g., averaged in windows).


%%%%%%%%%%%%%%%%%%
\subsection*{Tree sequences}

A tree sequence describes how a set of $n$ sampled chromosomes
are related to each other along a (linear) genome of length $L$ \citep{kelleher2016efficient}.
These trees are highly correlated,
a fact which allows them to be stored and processed very efficiently,
as described in \citet{kelleher2018efficient}.
This is encoded as a sequence of trees, $\treeseq = (T_1, T_2, \ldots,
T_{n(\treeseq)})$
and a sequence of breakpoints $0 = a_0 < a_1 < \cdots < a_{n(T)} = L$,
where $T_k$ describes the genealogical relationships of the samples
over the segment of genome betwen $a_{k-1}$ (inclusive) and $a_k$ (exclusive).
We say that tree $T_k$ \emph{covers} the (half-open) segment $[a_{k-1}, a_k)$,
and call the length of this segment its \emph{span}, denoted $L_k = a_k - a_{k-1}$.
We refer to the branches in each tree using the most recent node,
so for instance the branch describing a relationship where $u$ is the child of $v$
is associated with $u$.
The length of this branch -- i.e., the difference in birth times of $v$ and $u$ --
we denote $t_T(u)$ (dropping $T$ if it is clear from context).
Note that although the same parent-child relationship may exist across many adjacent trees
(this is called an ``edge''),
rearrangements of genealogical relationships
can cause the precise set of samples that lie below that edge to differ across
trees.


%%%%%%%%%%%%%%%%%%
\subsection*{Defining a statistic}

\jk{I think we're missing an overall overview of how to specify a statistic
and basic definitions of the framework itself. Before getting into the
specifics below, it would be good to give an more abstract idea of
what's happening. Perhaps with reference to a diagram?}

The basic ingredients that we use to summarize tree shape and genotype patterns
are a set of values associated with each sampled node, which we call the ``weights''.
For instance, we could calculate how many of a certain set $A$ of samples
are below a particular edge in a tree
by (a) assigning each of these samples weight 1 (and other nodes weight zero), then
(b) finding the total weight of all nodes in the subtree below that edge.
It turns out to be useful to allow more general weights (including negative ones).
These subtree-associated weights can be efficiently propagated along a tree sequence
because adjacent trees are simliar (and so common subtrees do not need to be updated).

We will use weights of the form described above to count numbers of samples in each subtree
frequently enough that we give them a special name --
for a subset set $A$ of samples,
the \emph{indicator weights of $A$} are the sample weights $\bone_A$ with
$\bone_A(u) = 1$ if $u \in A$ and $\bone_A(u) = 0$ otherwise.

\begin{definition}[Subtree weights]
    Given a list of \emph{sample weights} $w$ and a tree $T$,
    define the \emph{subtree weight} $\nw_T(u)$ for each node $u$ to be the sum of all sample weights
    of every node descending from (and including) $u$ in the tree:
    \begin{align*}
        \nw_T(u) = \sum_{v \,:\, v \le_T u} \iw(v) ,
    \end{align*}
    where $v \le_T u$ if $u$ is on the path from $v$ to root in $T$.
\end{definition}

We allow vector-valued weights,
i.e., $\iw(v)$ may be a vector $(\iw_1(v), \ldots, \iw_m(v))$,
so that when summarising we have access to more than one aspect of each subtree.
% Below, in section XXX, we describe how to efficiently
% maintain a list of the weights of the subtrees below each node
% as we iterate through the tree sequence.

Next, we will describe how to use these subtree weights,
passed through a ``summary function'' in various ways,
to summarize genetic variation and tree shapes across the tree sequence.

\begin{definition}[Summary function]
    Given a weight vector $(w_1, \dots, w_k)$, the summary function
    $f: \mathbb{R}^k \rightarrow \mathbb{R}^m$ computes a vector
    $(\sigma_1, \dots, \sigma_m)$ summarising these weights.
\end{definition}


\jk{This definition is pretty weak, but wanted something in there.}


%%%%%%%%%%%%%%%%%%
\subsection*{Node statistics}

Propagating indicator weights of a subset of samples up the tree
lets us easily count the number of those samples that inherit from each node in
the tree. Averaged across trees,
this tells us what proportion of the sample's genomes were inherited from each node.
Other choices of weights would tell us other information about the nodes.
Motivated by this, we define the
\textbf{node statistic} for node $u$
associated with summary function $f()$ and sample weights $\iw$
to be the sum of $f()$ applied to the weight of the subtree below node $u$,
averaged across the genome:
\begin{align}
    \node(f, \iw)_u
    =
    \frac{1}{L} \sum_{k=1}^{n(\treeseq)} L_k f(\nw_{T_k}(u)).
\end{align}
% i.e., the length of the segment of genome that the tree extends for.
% If this is computed over only a ``window'' of the genome
% then $L_k$ is replaced by the length of the portion of that window
% that $T$ extends for.

\begin{example}[Ancestry proportions] \label{ex:ancestry_props}
    If $\iw$ are the indicator weights of the set $S$ of $n$ samples,
    and then $\iw_u / n$ is the proportion of the samples in $S$ that lie below $u$.
    Therefore, if $f(x) = x / n$,
    then $\node(f, \iw)_u$ is the proportion of the genomes of $S$
    that are inherited from (ancestor) $u$.
\end{example}


%%%%%%%%%%%%%%%%%%
\subsection*{Site statistics}

The first method of summarizing uses weights
to compute a summary for each site in the genome.
These are often reported by averaging the values over windows.
The general definition uses the \emph{allele weights} at each site in the genome,
defined simply to be the total weight of all nodes who have inherited that allele.

\begin{definition}[Allele weights]
    The \emph{allele weight} for allele $a$ at site $j$ is the sum of the weights
    of all nodes inheriting this allele:
    \begin{align*}
        \aw_j(a) = \sum_{v \st g_j(v) = a} \iw(v) ,
    \end{align*}
    where $g_j(v)$ is the allele carried by node $v$ at site $j$.
\end{definition}

If there has been only one mutation at the site,
then $\aw_j(a)$ is equal to the subtree weight of the node that the mutation producing $a$
appeared above,
while more generally $\aw$ is not necessarily equal to a subtree weight
but is easily computable from them.

We then define the \textbf{site statistic} of site $j$ computed using a summary function $f()$
and a set of sample weights $\iw$
to be the sum of $f()$ applied to each of the allele weights:
\begin{align}
    \site(f, \iw)_j
    &=
    \sum_{a} f(\aw_j(a)) ,
\end{align}
where the sum is over all unique alleles at site $j$.

Of course, we often want to summarize statistics across regions of the genome (``windows'').
To do this, we overload notation somewhat and use a subscript $[i,j)$ to denote an average
over the corresponding portion of the genome:
\begin{align}
    \site(f, \iw)_{[i,j)}
    &=
    \frac{1}{j-i} \sum_{k=i}^{j-1} \site(f, \iw)_k
\end{align}
Recall that we have always defined the summary function $f()$ to be zero at monomorphic sites,
so that the sum will be only over polymorphic sites
(although the normalization is always by total number of sites, to allow comparison between regions).


\paragraph{What about polarization?}
In the definition above we simply sum over all alleles at each site.
However, sometimes it is useful to distinguish the \emph{ancestral} allele
(i.e., the allele at the root of the tree) from the remaining derived alleles.
This allows statistics in principle to differentiate ancestral from derived alleles,
information which is available in practice (albeit noisily).
One way to make use of this information is to sum over only derived alleles.
We say a site statistic is \textbf{polarized} if we do this,
and so
\begin{align} \label{eqn:site_unpolarized}
    \sitep(f, \iw)_j
    &=
    \sum_{a \in D_j} f(\aw_j(a)) ,
\end{align}
where $D_j$ denotes the set of all alleles at site $j$ except the allele at the root of the tree.
Note that this may not be quite what is expected,
for instance, if there has been a back mutation to the ancestral allele at some point in the tree,
or if there have been mutations to distinct alleles on different parts of the tree
such that the ancestral allele is no longer present.
However, since these situations depend on multiple mutations occurring at a single site,
they are relatively rare in practice.

\begin{example}[Nucleotide diversity] \label{ex:site_diversity}
    Suppose we want to compute the mean density of nucleotide differences
    in a group $S$ of $n$ sequences.
    To do this,
    % let $\iw_u = 1$ for $u \in S$ and $\iw_v = 0$ otherwise,
    let $\iw = \bone_S$,
    so that $\nw(u)$ gives the number of nodes in $S$ below $u$,
    and define
    \begin{align*}
        f(x) = \frac{x (n - x)}{n (n-1)} .
    \end{align*}
    Then $\site(f, \iw)_{[a,b)}$ is mean nucleotide diversity in the region between $a$ and $b$.
\end{example}

\begin{example}[Nucleotide divergence] \label{ex:site_divergence}
    Now suppose we want to compute the mean density of nucleotide differences
    \emph{between} two nonoverlapping groups of samples, $S_1$ and $S_2$,
    with $n_1$ and $n_2$ samples, respectively.
    As before,
    let $\iw_j = \bone_{S_j}$,
    so that $\nw_{j}(u)$ gives the number of nodes in $S_j$ below $u$.
    Then we define
    \begin{align*}
        f(x_1, x_2) = \frac{x_1 (n_2 - x_2)}{n_1 n_2} .
    \end{align*}
    Then $\site(f, \iw)_{[a,b)}$ is mean nucleotide divergence between the two groups
    in the region between $a$ and $b$.
\end{example}

\begin{example}[Phenotypic correlations] \label{ex:site_correlations}
    Suppose that for each sample $u$ we have a numeric phenotype, denoted $z(u)$,
    and we want to compute the correlation between this phenotype
    and the genotype at each site in the tree sequence.
    For convenience, suppose $z$ is normalized so that $\sum_u z(u) = 0$ and $\sum_u z(u)^2 = 1$.
    Then, if $g_j$ is a vector of binary genotypes (so $g_j(u) = 1$ if $u$ carries the derived allele),
    then the covariance of $z$ with $g_j$ is just $\sum_u z(u) g_j(u) = \sum_{u \st g_j(u) = 1} z(u)$,
    i.e., the sum of the phenotypes of all samples carrying the derived allele,
    divided $\sqrt{n - 1}$, where $n$ is the number of samples.
    Since the phenotypes sum to zero, this is also equal to 
    $- \sum_{u \st g_j(u) = 0} z(u)$.
    Therefore, if we write $p_j = \sum_u g_j(u)$ for the derived allele frequency,
    we can write the squared correlation as the sum across alleles as
    \begin{align*}
        r_j^2 =
        \frac{\left( \sum_{u \st g_j(u) = 0} z(u)\right)^2}{2p(1-p)n(n-1)} 
        + \frac{\left( \sum_{u \st g_j(u) = 1} z(u)\right)^2}{2p(1-p)n(n-1)}  .
    \end{align*}
    We can compute this as a site statistic by defining $\iw_{1}(u) = z(u)$, and $\iw_{2}(u) = 1/n$,
    and $f(x_1, x_2) = x_1^2 / (2 x_2 (1 - x_2) n (n-1))$:
    then $\site(\iw, f)_j = r_j^2$ is the squared correlation between $z$ and the allele at site $j$.
\end{example}

\textbf{Note:} computing correlations with branches instead of SNPs is like stuff done in \citet{zollner2005coalescent} and \citet{minichiello2006mapping}, maybe.

\begin{example}[Patterson's $f_4$] \label{ex:site_f4}
    Given four disjoint groups of samples, $S_1$, $S_2$, $S_3$, and $S_4$,
    Patterson's $f_4(S_1, S_2; S_3, S_4)$ statistic for an allele with frequency $p_i$ in group $S_i$
    is $(p_1 - p_2)(p_3 - p_4)$ (and then this is typically averaged over sites).
    To rewrite this as a sum over alleles, note that
    $p_1 - p_2 = p_1 (1 - p_2) - (1 - p_1) p_2$,
    and so the statistic counts with positive weight
    alleles that split $S_1$ and $S_3$ from $S_2$ and $S_4$,
    and negative weight ones that split $S_1$ and $S_4$ from $S_2$ and $S_3$.
    Therefore, if as before we
    let $\iw_j = \bone_{S_j}$
    % let $\iw_{ju} = 1$ if $u \in S_j$
    (so that $\iw_{ju}$ tells us the number of samples in $S_j$ descended from $u$),
    and write $n_i$ for the number of samples in $S_i$,
    \begin{align*}
        f(x_1, x_2, x_3, x_4)
        =
        \frac{x_1}{n_1}
        \frac{x_3}{n_3}
        \left(1 - \frac{x_2}{n_2}\right)
        \left(1 - \frac{x_4}{n_4}\right)
        -
        \left(1 - \frac{x_1}{n_1}\right)
        \left(1 - \frac{x_4}{n_4}\right)
        \frac{x_2}{n_2}
        \frac{x_3}{n_3}
    \end{align*}
    then $\site(\iw, f)_{[a,b)}$ is equal to Patterson's $f_4(S_1, S_2; S_3, S_4)$ statistic
    across the region of the genome between $a$ and $b$.
\end{example}

\paragraph{Polyallelic sites}
The statistics defined above map exactly onto those standard in population genetics
when all sites are biallelic.
If there is a consensus in the field
for how to use sites with more than two alleles,
it is to reduce to biallelic data, by either discarding other sites
or by marking all non-ancestral alleles as (the same) ``derived'' allele.
We choose to define statistics in a way that still makes formal sense for polyallelic sites,
so that for instance a site with ancestral allele $A$ and derived alleles $C$ and $T$
would have site statistic $f(\aw(C)) + f(\aw(T))$.
This is a natural defintion in that it agrees with what you'd get
from looking at haplotype differences.
For instance, the definition of nucleotide divergence in Example \ref{ex:site_divergence}
gives the probability that a randomly chosen sample from each set differ.
This definition makes sense even at polyallelic sites,
and (as we discuss below) has a natural correspondence to statistics related to branch lengths.

\plr{Figure: example of propagating weights and labeling nodes with weights,
    in the case that weights are number of samples you are ancestor of.
    Also, this is a node statistic.}

\plr{Figure: example of branch and site statistics}


%%%%%%%%%%%%%%%%%%
\subsection*{Branch statistics}

Genetic variation is informative about many processes
precisely because it tells us about the underlying patterns of genealogical relatedness.
In other words, the genomes are useful in so far as they tell us about the trees.
In the case where we actually have the trees, or a good proxy for them,
it is natural to summarize them directly, rather than working indirectly with the genotypes.
If we assume that no two mutations occur at the same genomic position --
i.e., the \emph{infinite sites model} --
then there is a natural correspondence between summaries of genotypes and summaries of tree shape.
If mutations occur at a constant rate in time and along the genome,
then the \emph{expected} number of mutations that occur somewhere along a branch of a tree
over some segment of genome
is equal to the mutation rate multiplied by the length of the segment and by the length of the branch.
In other words, the ``area'' of a branch in a tree of the tree sequence,
defined as its span (right minus left endpoint) multiplied by its length (parent time minus child time)
is equal to the expected number of mutations that will land on it, up to the mutation rate.
If the mutation rate is constant,
then this gives us an isometry --
measuring tree distances in branch lengths versus in numbers of mutations
are equal in expectation, up to a constant.

This makes it natural to define
a statistic of tree shape by summing these expected contributions across its branches.
Concretely, the \textbf{branch statistic} of a tree $T$
obtained from summary function $f()$
and sample weights $\iw$ with total weight $\tiw = \sum_u \iw(u)$
is defined to be
\begin{align}
    \branch(f, \iw)_T
    &=
    \sum_{k=1}^{n(\treeseq)} L_k \left\{ f(\nw_{T_k}(u)) + f(\tiw - \nw_{T_k}(u)) \right\}  ,
\end{align}
where $t_u$ is the length of the branch above node $u$ in the tree $T$
and $\nw(u)$ is the total weight of the subtree below $u$ (as defined above).
The term $\tiw - \nw_{T_k}(u)$ gives the total weight \emph{not} in the subtree of $T_k$ below $u$,
because the sum of all weights in the tree is always constant,
and so $\sum_u \nw_T(u) = \tiw$.
The value $f(\nw(u))$ is the summary value that would be added to a site statistic
if a single mutation occurred on the branch above $u$,
and $f(\tiw - \nw_{T_k}(u))$ is the value that would be added due to its complementary allele,
so $\branch(f, \iw)_T$ gives the expected contribution of this branch to $\site(f, \iw)$,
per unit of sequence length that the tree $T$ covers.

In practice it is probably most useful to average branch statistics
over a region of the genome,
which we do by averaging the tree statistics over the region
with weights proportional to the trees' spans.
We write this with similar notation:
\begin{align}
    \branch(f, \iw)_{[a,b)}
    &=
    \frac{1}{b-a} \sum_T \ell_T(a,b) \branch(f, \iw)_T ,
\end{align}
where $\ell_T(a,b)$ is the length of the region in $[a,b)$ that the tree $T$ extends for
(i.e., if $T$ covers the half-open interval $[c,d)$,
then $\ell_T(a,b) = \max(0, \min(b,d) - \max(a,c))$).

The ``unpolarized'' version of this statistic, corresponding to equation \ref{eqn:site_unpolarized},
omits the second term:
\begin{align} \label{eqn:branch_unpolarized}
    \branch_+(f, \iw)_T
    &=
    \sum_{k=1}^{n(T)} L_k f(\nw_{T_k}(u)) .
\end{align}
However, the correspondence between the unpolarized statistics is less tight
than for the polarized ones.


\begin{example}[Mean TMRCA] \label{ex:branch_diversity}
    If we take $\iw$ and $f$ exactly as in the ``Nucleotide diversity'' example above,
    then $f(u)$ gives the probability that the branch above $u$
    lies on the path from two randomly chosen samples from $S$
    on the path up to their most recent common ancestor (MRCA).
    Therefore, $\bar \branch(f, \iw)$ for these choices
    gives the mean total distance in the tree between two samples from $S$,
    averaged across the sequence.
    (This is twice the TMRCA if the samples are all from the same time.)
\end{example}

\begin{example}[Phenotypic correlation with pedigree] \label{ex:branch_correlation}
    If we take $\iw$ and $f$ as in the ``Phenotypic correlations'' example above,
    what does $\branch(f, \iw)$ tell us?
    The statistic tells us the \emph{expected} correlation between phenotype and any mutations
    appearing on the tree, which is a summary of how much local relatedness
    aligns with similarity in phenotype.
\end{example}


\begin{example}[Patterson's $f_4$] \label{ex:branch_f4}
    Suppose that the four subsets each consist of only a single sample.
    The summary function $f(x_1, x_2, x_3, x_4)$ for the $f_4$ statistic
    then assigns weight 1 to any branch that separates $x_1$ and $x_3$ from $x_2$ and $x_4$,
    and weight -1 to any branch that separates $x_1$ and $x_4$ from $x_2$ and $x_3$.
    The statistic $\branch(f, \iw)$ therefore
    gives the difference in averages of these two types of branch,
    averaged over choices of samples from the sample sets and averaged across the genome.
\end{example}

%%%%%%%%%%%%%%%%%%
\section*{Duality of site and branch statistics}

Under a neutral, infinite-sites model of mutation with constant mutation rate across time,
the expected number of mutations per branch is proportional to its length.
This implies an isormorphism between ``site'' and ``branch'' statistics defined above,
which is discussed in more detail in \citet{ralph2019empirical}.
For instance, the site statistic of Example \ref{ex:site_diversity} (genetic diversity)
and the branch statistic of Example \ref{ex:branch_diversity} (mean TMRCA)
use the same summary function $f(x) = x(n-x)/n(n-1)$.
These are closely related because under an infinite-sites model of mutation,
two sequences differ at a site only if there has been a mutation somewhere on the branches going back
to their most recent common ancestor,
and so if mutations occur with constant rate,
the expected value of genetic diversity,
averaging over mutational noise give the tree sequence,
is equal to the mutation rate multiplied by the average distance between the two in the trees.

This relationship is true more generally.
In fact, for any region of the genome between $a$ and $b$,
\begin{align}
    \branch(f, \iw)_{[a,b)}
    =
    \E\left[ \site(f, \iw)_{[a,b)} \right] ,
\end{align}
where the expected value averages over infinite-sites mutations with rate 1
(and so lengths are measured in units of expected numbers of mutations).
Let's unpack this statement a bit more: what exactly is the mutational model?
First, we are taking the mutation rate to be 1, i.e.,
the expected number of mutations that occur on a region of the genome of length $\ell$
over $t$ units of time is equal to $\ell t$.
This just amounts to a change of units --
for instance, if genome length is measured in nucleotides,
and the probability of mutation is $\mu$ per nucleotide and per generation,
then we are measuring times in units of $1/\mu$ generations.
Second, we are assuming that the probability of per-site mutation is low enough
that no two mutations occur at the same site
-- the fact that they do, occasionally, means that this is an approximation.
Third, we are assuming that mutation rates are constant through time and across the genome.
Of course, the statement remains true if we can measure distance along the genome and time
in a way that mutation rates are constant, but how these vary is generally unknown.

Note that the expected product of two site statistics,
$\E[\site(f, \iw) \site(g, \iw')]$
is not equal to $\branch(f, \iw) \branch(g, \iw')$,
because they are not independent.
However, it is always possible to define a branch statistic that
gives the expected value of the product.
How to do this is described in \citet{ralph2019empirical}.

This correspondance is useful for several things:
(1) removing noise of mutations
(2) understanding what statistics tell us about genealogies

\paragraph{Example: a selective sweep}
In this view,
site statistics are noisy approximations to the corresponding branch statistic
-- but, how noisy?
How big is the contribution of mutation to the overall sampling variance of a statistic?
Figure~\ref{fig:sweep_duality} shows this in a particular situation:
diversity along the chromosome following a few selective sweeps (two full sweeps and three partial).
The top two plots compare ``branch'' diversity -- i.e., as computed only with tree shape --
to ``site'' diversity computed from sequence generated by 20 independent assignments of mutations to the same tree sequence,
with mutation rates $10^{-9}$ and $10^{-8}$, respectively.
We see that as the mutation rate increases, the signal of decrease in diversity around swept loci becomes more clear,
and approaches ``branch'' diversity.
These were computed using the entire population of 1,000 individuals;
how does sampling variance contribute?
Not much, it turns out -- the bottom plot shows both site and branch diversity
computed from 20 nonoverlapping groups of 100 samples.
Neither site or branch diversity vary much between these samples,
implying that the subsample gives us a good estimate of the whole-population values of each.
However, as we see in the top figure,
whole-population site diversity is itself only a quite noisy estimator of branch diversity.


\begin{figure}
    \includegraphics{{figures/swept.1000.999.1e-09.diversity}.pdf}
    \caption{
        Mean genetic diversity and time to most recent common ancestor
        in 500Kb windows along a 50Mb genome (0.5 Morgans) following several selective sweeps.
        In each case, ``site'' is mean genetic diversity (Tajima's $\pi$) divided by mutation rate,
        and ``branch'' is the corresponding branch statistic described above.
        The tree sequence was produced by simulating mutations under positive selection with mutation rate $10^{-12}$ 
        in a population of size 1000 for 400 generations using SLiM,
        followed by recapitation with $N_e=1000$.
        The selected alleles at the marked sites have selection coefficients between 0.08 and 0.25,
        and are at frequencies 96.8\%, 100\%, 16.25\%, 100\%, and 82.6\% in the final generation, respectively.
        All curves use the same tree sequence, including selected mutations,
        and with additional neutral mutations added.
        \textbf{(top)}
        Diversity within the entire population,
        computed using for mode ``site'' from 20 independent assignments of mutations to the same tree sequence with mutation rate $\mu = 10^{-9}$.
        \textbf{(middle)}
        As in the top panel, but with mutation rate $\mu=10^{-8}$,
        showing that as mutation rate increases, mode ``site'' (divided by $\mu$) converges to mode ``branch''.
        \textbf{(bottom)}
        ``Site'' and ``branch'' diversity within 20 disjoint samples of size 100 each,
        computed on a single tree sequence with mutation rate $\mu = 10^{-9}$.
        \label{fig:sweep_duality}
    }
\end{figure}



%%%%%%%%%%%%%%%%%%
\subsection*{Separating mutational from genealogical noise}

\plr{figure showing noise of some site statistic decreasing and approaching the branch statistic
    as the mutation rate increases}


%%%%%%%%%%%%%%%%%%
\section*{Efficient algorithms and implementation}

%%%%%%%%%%%%%%%%%%%%%
\subsection*{Algorithms}
write-up of algorithm for maintaining branch statistic across trees

%%%%%%%%%%%%%%%%%%%%%
\subsection*{Speed}

compare speed to scikit-allel


%%%%%%%%%%%%%%%%%%%%%%%
\section{misc stuff}


%%%%%%%%%%%%%%%%%%%%%
\subsection*{Summary}
We have defined three types of statistic using the same weighting scheme:
\begin{enumerate}
    \item \textbf{Site statistics}
        -- summarizes weights below each allele at each polymorphic site.
    \item \textbf{Node statistics}
        -- summarizes total weight below each node.
    \item \textbf{Branch statistics}
        -- summarizes total weight below each branch, multiplied by branch length.
\end{enumerate}


\plr{TO-DO:}
\begin{itemize}

    \item What about normalization?
        In the initial implementation, Node statistics were divided by the length
        for which the node was an ancestor to any sample, rather than just total sequence length
        To make these more analogous to the other stats, we think we want to remove this.
        The denominator for the original case is easily computable, by using
        $f(x) = 1$ if $x>0$ and $f(x)=0$ otherwise, with appropriate (positive) weights.

\end{itemize}


%%%%%%%%%%%%%%%%%%
\subsection*{Tree sequences}

Quick review of tree sequence terminology
and storage:
nodes,
edges,
sites,
mutations.


%%%%%%%%%%%%%%%%%%
\subsection*{Algorithms for fast computation}



%%%%%%%%%%%%%%%%%%
\section*{Case studies}
% AKA "examples"


%%%%%%%%%%%%%%%%%%
\subsection*{Ancestry proportions}


%%%%%%%%%%%%%%%%%%
\subsection*{Population genetics statistics}


%%%%%%%%%%%%%%%%%%
\subsection*{Genomic PCA}

Let $G$ denote the genotype matrix, so that $G_{ja} \in \{0,1\}$
is the genotype of the $a^\text{th}$ individual
at the $j^\text{th}$ site, with `1` denoting the derived state
(we assume here biallelic markers).
Also let $P_j = (1/n) \sum_a G{ja}$ denote the vector of allele frequencies.
Then the \emph{genetic covariance} between samples $a$ and $b$ is
\begin{align*}
    C_{ab}
        &= \frac{1}{L} \sum_j (G_{ja} - P_j) (G_{jb} - P_j) . %  \\
        % &= \frac{1}{L} \left( (G - P \bone^T)^T (G - P \bone^T) \right)_{ab} .
\end{align*}
This can be computed by letting
$\iw_a = \delta_a$ and $\iw_b = \delta_b$
(these weights mark nodes above $a$ and $b$ respectively)
and $\iw_t = \bone / n$
(this weight gives the proportion of samples below each node).
Then, with
$f(\nw_a, \nw_b, \nw_t) = (\nw_a - \nw_t) (\nw_b - \nw_t)$,
we can compute the covariance as
\begin{align*}
    C_{ab} = \frac{1}{2} \bar \site(f, (\iw_a, \iw_b, \iw_t)) .
\end{align*}
This follows, including the factor of two,
because the summary function applied to the ancestral allele
at a site with derived allele frequencies $p_a$, $p_b$, and $p_t$
in sample $a$, sample $b$, and total, respectively, is
$f(1 - p_a, 1 - p_b, 1 - p_t) = f(p_a, p_b, p_t)$.

Computing the full covariance matrix for a large number of samples may be infeasible.
However, we can efficiently find $y^T C z$ for arbitrary vectors $y$ and $z$.
This works because if we set the initial weights to $y$,
then each allele's weight is equal to the sum of the entries of $y$
corresponding to samples carrying that allele.
Then, if we define
$f_{yz}(u, v, t) = (u - (\sum_a y_a) t) (v - (\sum_b z_b) t)$,
then
\begin{align} \label{eqn:yCz}
    \frac{1}{2} \bar \site(f_{yz}, (y, z, \iw_t))
        &= \frac{1}{L} \sum_j (\sum_a y_a G_{ja} - (\sum_a y_a) P_j)
                    (\sum_b z_b G_{jb} - (\sum_b z_b) P_j) \\
        &= y^T C z .
\end{align}

\plr{Hmph: to actually compute $Cz$ we need one weight vector for each sample, again.
    TODO: look up if you can do Krylov just by computing $y^t C z$ for a reasonable number of vectors.
    We could compute $Cz$ quickly if we could propagate down the tree:
    we'd have $(Cz)_a$ equal to the sum of the weights at all nodes \emph{above} $a$. }

Where does the signal from an eigenvector come from?
Let $y$ be an eigenvector of $C$ with eigenvalue $\lambda$, normalized so that $\|y\|^2 = 1$.
Since $Cy = \lambda y$,
we can write $1 = \sum_a y_a^2 = (1/\lambda) y^T C y$.
However, equation \eqref{eqn:yCz} gives $y^T C y$ as a sum over \emph{sites},
and thus distributes the variance in $G$ associated with $y$
across the tree sequence.
(This is true because of the singular value decomposition of $G$.)



%%%%%%%%%%%%%%%%%%
\subsection*{Visualizing the time profile of PCA}



%%%%%%%%%%%%%%%%%%%%%%%
\section*{Discussion}

In this paper,
we consider only so-called \emph{single site} statistics
(which we will define concretely shortly).
Extensions to statistics involving the pairwise joint distribution of genotypes across sites,
such as linkage disequlibrium,
are straightforward, and are planned for future work.
Haplotype-based statistics may require a different class of algorithms.


\paragraph{Diploid statistics}
The statistics we develop here are in essence \emph{haploid} statistics.
These tools can be used to compute a statistic about polyploid individuals,
but they do not scale to large numbers of individuals.
For instance, to compute the heterozygosity of a set of samples
-- this is $(1/N) \sum_{i=1}^N 2 p_i (1-p_i)$, where $p_i$ is the derived allele frequency
in the $i^\text{th}$ individual --
one would need to have an entry in the weight vector for \emph{each individual}.
The genotype of a diploid at a site with a single mutation that occurred on the branch above node $n$
is determined by where that node falls relative to the diploid's two haplotypes in the tree:
if $n$ is on the path from either haplotype to their MRCA, the individual is heterozygous;
if $n$ is on the path from the MRCA to the root, the individual is homozygous derived;
and the individual is homozygous ancestral otherwise.
This suggests the following generalization of our current scheme for diploids:
for a list $(I_1, \ldots, I_N)$ of pairs of nodes,
and two vectors $w^{(1)}$ and $w^{(2)}$ of weights of length $N$,
add weight $w^{(1)}_i$ to each node on the path between the two nodes of $I_i$,
and weight $w^{(2)}_i$ to each node on the path from the MRCA of $I_i$ to the root.
With $k$ such weighting schemes, we would then need the summary function $F$
to take $k \times 2$ values as input.
Efficiently updating these ``diploid'' weights on the nodes should be possible,
but is not a straightforward extension of our current method.




%%%%%%%%%%%%%%%%%%%%%%%
\subsection*{Acknowledgements}
Thanks to Graham Coop for useful suggestions.


\bibliography{references}

\newpage
\appendix

{\Large \plr{OLD STUFF}}

%%%%%%%%%%%%%%%%%%
\subsection*{Efficient computation}

First we describe how to define
and efficiently compute a quite general set of functions of a collection of sequences
using tools from tree sequences.
In the next section we will specialize these to a natural set of statistics.

\plr{Since the ``weighting function' can take negtive values, maybe ``weight'' is not a good term.  Instead?}

\paragraph{Weighted site averages}
Suppose that we have $k$ groups of samples, $A_1$, \ldots, $A_k$.
% with $n_1$, \ldots, $n_k$ individuals in each, respectively.
The \emph{weight} we compute for a given site is a sum of weights computed for each allele at that site,
and the weights are found as a function of the vector of numbers of individuals in each group inheriting that allele.
Concretely, we specify a \emph{weighting function} $w(x_1, \ldots, x_k)$,
and we say that the \emph{average site weight} of a set of $L$ sites is
\begin{align} \label{eqn:average_site_weight}
    \site_w := \frac{1}{L} \sum_{i=1}^L \sum_{s \in I_i} w(x_1(s), \ldots, x_k(s)) ,
\end{align}
where $I_i$ is the set of alleles seen at that site and
$x_j(s)$ is the number of individuals in $A_j$ that carry allele $s$.

Before we describe how to compute this,
we give the analogous topological quantity,
that is equal to the expected weighted site average
under a continuum-sites model.

\paragraph{Weighted branch length averages}
The \emph{average branch length weight}
is computed almost just as above:
\begin{align} \label{eqn:average_branch_weight}
    \branch_w :=  \frac{1}{L} \sum_{i=1}^L \sum_{e \in T_i} \ell(e) w(x_1(e), \ldots, x_k(e)) ,
\end{align}
except that now $T_i$ is the genealogical tree at site $i$,
the sum is over edges $e$ in the tree, $\ell(e)$ is the length of the edge $e$,
and $x_j(e)$ is the number of individuals in $A_j$ that fall below $e$.

\emph{Note:} somewhat more generally,
we could assign a set of weights to the individuals in each group,
and let $x_j$ denote the total weight of the individuals in $A_j$
inheriting from that allele or edge.
For simplicity we do not pursue this here.
\plr{this is a good use of the word ``weight'' (although these could also be negative)}


\paragraph{Computation}
The \emph{tree differences} encoded by a tree sequence allow us to efficiently update the quantities $x$ used above.
As it is more straightforward, we first describe the algorithm
to compute the branch length version.

\plr{Rework these into three algorithms: one to maintain the sample weights $\iw$
    (basically already described as `tracked nodes' in msprime)
    and the other two that use this.}

\paragraph{Total branch length weight algorithm}
In a tree sequence, every branch in any tree is associated with its terminal \emph{node}.
We will keep updated an array $x$ so that for any node $e$,
the value $x[e][j]$ is the number of samples in $A_j$ below node $e$ in the current tree.
This will be zero for any node \emph{not} appearing in the current tree.
We also maintain $T$, the total weight of the current tree.
\begin{itemize}
    \item \emph{Initialize} all elements of $x$ to zero
        except that $x[a][j] = 1$ for each $a \in A_j$ and $1 \le j \le k$.
        Also set $S=0$ and $T=0$.
    \item Pop the next \emph{tree difference} off the stack: an interval with endpoints $\ell < r$,
        a set of edges $\mathcal{R}$ to remove, and a set of edges $\mathcal{A}$ to add.
    \item \emph{Remove} each edge (parent, child) in $\mathcal{R}$ from the tree,
        let $u$ = parent, and while $u != -1$,
        subtract $x[child]$ from $x[u]$ and set $u$ to be parent($u$).
    \item \emph{Add} each edge (parent, child) in $\mathcal{R}$ to the tree,
        let $u$ = parent, and while $u != -1$,
        add $x[child]$ to $x[u]$ and set $u$ to be parent($u$).
    \item For each node that has changed, let $w_\text{old}$ and $w_\text{new}$
        be the weights before and after the change in $x$,
        and add $w_\text{new} - w_text{old}$ to $T$.
    \item Add $T \times (r-\ell)$ to the running total, $S$.
    \item Return to (2).
\end{itemize}
\plr{should modify code to maintain a vector of $w$ values}
The average weighted branch length is obtained by dividing by the total number of sites.

The algorithm for sites is similar, also maintaining the array $x$,
and making use of the association of each mutation with a particular node in the tree.

\paragraph{Total site weight algorithm}
This works as the \emph{total branch length weight algorithm} except that steps (5--6) are replaced with:
\plr{gah this is terrible}
\begin{itemize}
    \item[5'] \emph{Find allele weights} for each mutated site in the tree:
        if $a$ is the ancestral allele,
        set $u[a]$ to be the value of $x$ at the root; and then,
        for each mutation at node $e$ with derived allele $b$
        and parent mutation occurring at node $p$ with derived allele $c$,
        add $x[e]$ to $u[b]$, and subtract $x[p]$ from $u[c]$.
    \item[6'] For each unique allele $a$ seen,
        add $w(v[a])$ to $T$.
\end{itemize}
\plr{check that algorithm deals correctly with more than one mut per node}

%%%%%%%%%%%%%%%%%%
\subsection*{General properties}

Here we describe some properties that we would like our statistics to have.

\paragraph{They do not depend on choice of reference allele.}
This is achieved by applying the \emph{same} weighting function to every allele.
Ancestral information can be incorporated by including an outgroup or ancestral sequence
as an additional group of size one.
Equivalently, we always work with \emph{unrooted} trees.

\paragraph{They do not depend on sample size.}
We'd like our statistics to be \emph{consistent estimators}
of a single quantity, and should therefore not depend on sample size in expectation.
\plr{insert this above}
This is achieved by always computing probabilies of sampling \emph{without replacement},
i.e., replacing each occurence of $p^k$ with $(x)_k/(n)_k$.
This lets us compare values of a given statistic
that are computed on different data sets with different sample sizes.

\paragraph{Equivalence of ``site'' and ``branch length'' statistics}
The definitions above are made so that for any weighting function,
$\site_w$ is a moment estimator of $\branch_w$, i.e.,
that $\E[\site_w] = \branch_w$ for \plr{somehow say continuum of sites here}.

%%%%%%%%%%%
\appendix

\section{Linear regression}

Let $h$ be a trait, $Z$ be a matrix of covariates, and $g$ be a vector denoting inheritance
(so, with $n$ samples, $h$ and $g$ are both $n$-vectors and $Z$ is an $n \times k$ matrix).
We would like to find the coefficient of $g$ in the linear regression of $h$ against $g$ and $Z$,
without doing full multivariate regression for every new $g$,
using the fact that $Z$ is always the same.
Suppose that $Z^T Z = I$ and that the vector of all ones is in the span of the columns of $Z$,
although in the implementation we post-process $Z$ to make this the case.
Then, let $a$ be the number and $b$ be the $k$-vector satisfying
\begin{align*}
    a, b = \text{argmin}\left\{ \sum_i \left( w_i - a g_i - \sum_j Z_{ij} b_j \right)^2 \right\}
\end{align*}
Writing this in block matrix notation, $a$ and $b$ minimise
\begin{align*}
    \left\| 
        \left[ \begin{array}{@{}c@{}} h \end{array}\right]
            - 
        \left[ \begin{array}{@{}c|c@{}} g & Z \end{array} \right]
            \left[ \begin{array}{@{}c@{}} a \\ \hline b \end{array} \right]
    \right\|^2 .
\end{align*}


Letting $B = [g | Z]$, the solution to this is
\begin{align*}
    \left[\begin{array}{@{}c@{}}a\\\hline b\end{array}\right] 
        = (B^T B)^{-1} B^T h ,
\end{align*}
as long as $B^T B$ is invertible (which we assume to be the case).
Let $m = g^T g$ be the number of alleles in the sample coded 1,
let $u = g^T Z$ be the vector giving sums of the covariates of all samples carrying the allele,
and
\begin{align*}
    \alpha
    &=
        (g^T g - g^T Z (Z^T Z)^{-1} Z^T g) \\
    &=
        m - \sum_j u_j^2 .
\end{align*}
$\alpha$ is the norm of the component of $g$ not in the subspace spanned by the columns of $Z$,
so if $\alpha = 0$ then we want to return $a=0$.

Otherwise, by the inversion formula for a block two-by-two matrix,
since we have assumed that that $Z^T Z = I$,
\begin{align*}
    (B^T B)^{-1}
    &=
    \left[
        \begin{array}{@{}cc@{}}
            m & u \\
            u^T & I
        \end{array}
    \right]^{-1} \\
    &=
    \left[
        \begin{array}{@{}cc@{}}
            1/\alpha
            &
            - u / \alpha
            \\
            - u^T / \alpha
            &
            m \left(m I - u u^T \right)^{-1}
        \end{array}
    \right] .
\end{align*}
Now, the regression coefficient we seek is,
with $h_g = g^T h$,
\begin{align*}
    a
    &=
    \frac{1}{\alpha} \left(
        h_g - u Z^T h
    \right) .
\end{align*}

To compute this in the framework above,
first add a column of 1s to the covariates $Z$,
then decorrelate the resulting matrix, so that now $Z^T Z = I$.
Then, put this normalised version of $Z$ 
into the first $k$ columns of the weight matrix (so that $w_j(u) = Z_{uj}$),
set the $(k+1)^\text{st}$ column to the trait (so that $w_{k+1}(u) = w_u$),
and the final column to all $1$s (so $w_{k+2}(u) = 1$).
Also let $Z^T h = v$ be precomputed.
Then the sum of the traits of samples with the focal genotype is $h_g = x_{k+1}$,
and the allele count is $m = x_{k+2}$,
so that
\begin{align*}
    a
    &=
    \frac{
        h_g - \sum_{j=1}^k x_j v_j
    }{
        m - \sum_{j=1}^k x_j^2 } .
\end{align*}

In practice we square this and divide by two,
so that for biallelic loci the two alleles contribute an equal amount.
For loci with more than two alleles,
it would be more satisfying to return the proportion of variance in the trait
that is explained by \emph{all} of the alleles;
however, this would be more involved
(it would entail inversion of a $3 \times 3$ matrix for each locus).

% # NUMERICAL CHECK
% n <- 20
% k <- 5
% h <- rnorm(n)
% g <- rbinom(n, size=1, prob=0.5)
% oZ <- cbind(matrix(rnorm(n*k), nrow=n), rep(1,n))
% colnames(oZ) <- c(paste0("oZ", 1:k), "1")
% Z <- oZ %*% solve(chol(crossprod(oZ)))
% colnames(Z) <- paste0("Z", 1:(k+1))
% lm_a <- coef(lm(h ~ g + oZ))[2]
% hg <- sum(g*h)
% hZ <- crossprod(Z, h)
% u <- crossprod(Z, g)
% p <- sum(g)
% alpha <- (p - sum(u^2))
% a <- (1/alpha) * (hg - crossprod(u, hZ))
% 
% # check formula for regression coefs
% # and that coefficient of g doesn't change on linear transform of Z
% B <- cbind(g, Z)
% H <- cbind(solve(crossprod(B), crossprod(B, h)),
%            coef(lm(h ~ g + Z + 0)),
%            coef(lm(h ~ g + oZ + 0)))
% stopifnot(all(abs(diff(H[1,])) < 2e-15))
% # check for alpha, an other bits
% stopifnot(all(abs(crossprod(B, h) - c(hg, hZ)) < 2e-15))
% stopifnot(all(abs(- (1/alpha) * u - solve(crossprod(B))[1,-1]) < 2e-15))
% stopifnot(abs(solve(crossprod(B))[1,1] - (1/alpha)) < 2e-15)
% stopifnot(all(abs(p*solve(p*diag(k+1) - tcrossprod(u)) - solve(crossprod(B))[-1,-1]) < 2e-15))
% # and the final answer
% stopifnot(abs(a - lm_a) < 2e-15)


\end{document}
